\documentclass{article}
\usepackage{xeCJK}

\setCJKmainfont{Microsoft YaHei}
\linespread{1.5}

\begin{document}
一:\\
算法:\\
1. 按活动结束时间递增排序所有活动;\\
2. 将最后一个活动加入优化解中;\\
3. 从后向前遍历所有活动,选择与优化解集合兼容的、开始时间最晚的
活动加入优化解集合中;\\
4. 遍历结束,所得优化解集合即为最大相容集合。\\
证明:\\
实际上述算法是原算法反向执行的结果,因此,原算法与上述算法等价,其
总能产生最优解。

\newpage
二:\\
1)
在当天价格高于前一天价格时进行一次交易,即前一天购入,当天卖出,否则不交易。 \\
设 $S_i$ 为前 $i$ 天的最大收益 \\
1. 记 $S_0 = 0$;\\
2. 遍历 $prices$ 数组,若 $prices[i] > prices[i-1]$,则 $S_i = S_{i-1} 
+ prices[i] - prices[i - 1]$,,否则 $S_i = S_{i - 1}$;\\
3. 遍历完成, $S_n$ 即为所求最大收益。 \\
总计遍历一次数组,时间复杂度为 $O(n)$. \\
2) 
在只允许两次的情况下,贪心算法无法保证收益最大化。\\
贪心算法并非模拟真实的交易流程。\\
反例数组:$prices = [7, 1, 5, 3, 6, 4, 8]$
两次交易情况下,贪心算法结果仍为 $7$,而实际最大收益为 $8$.

\newpage
三:\\
在每次跳跃时,计算一次跳跃落点的下一步可以到达的最远位置,贪心地选择最远位置
更远的落点,依次下去直到到达最后一个点即可。\\


\end{document}