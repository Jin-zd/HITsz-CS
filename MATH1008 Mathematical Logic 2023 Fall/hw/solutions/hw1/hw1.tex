\documentclass{article}
\usepackage{xeCJK}
\usepackage{amsmath}
\usepackage{geometry}


\setCJKmainfont{Microsoft YaHei}
\linespread{1.5}
\setlength{\parindent}{0pt}
\geometry{a4paper,scale=0.75}

\begin{document}
1.\\
(1) 设 $p$ 为大学里的本科生,$q$ 为大学里的研究生:
\[ (p \wedge \neg q) \vee (q \wedge \neg p) \]
(2) 设 $p$ 为接到超速罚单, $q$ 为车速超过每小时 100 公里:
\[ p \to q\]
(3) 设 $p$ 为年满 18 周岁, $q$ 为有选举权:
\[
\neg p \to \neg q 
\]
2.\\
(2)对任意指派 $v$,有
\[
\begin{aligned}
    (\neg A \to \neg B)^v & = 1 - (\neg A)^v + (\neg A)^v \cdot (\neg B) \\
                          & = 1 - (1 - A^v) + (1 - A^v) \cdot (1 - B^v) \\
                          & = 1 - B^v + B^v \cdot A^v \\
                          & = (B \to A)^v
\end{aligned}    
\]
故该逻辑等价成立。\\
(4)对任意指派 $v$, 有
\[
\begin{aligned}
    (A \to (B \to C))^v & = 1 - A^v + A^v \cdot (B \to C)^v \\
                        & = 1 - A^v + A^v \cdot (1 - B^v + B^v \cdot C^v) \\
                        & = 1 - A^v \cdot B^v + (A^v \cdot B^v) \cdot C^v \\
                        & = 1 - (A \wedge B)^v + (A \wedge B)^v \cdot C^v \\
                        & = (A \wedge B \to C)^v
\end{aligned}
\]
故该逻辑等价成立。\\
(6) 对任意指派 $v$,有
\[
\begin{aligned}
    1 & = (\neg A \vee B)^v \\
      & = (\neg A)^v + B^v - (\neg A)^v \cdot B^v \\
      & = 1 - A^v + B^v -  (1 - A^v) \cdot B^v \\
      & = 1 -  A^v + A^v \cdot B^v
\end{aligned}
\]
即
\begin{equation}
    A^v \cdot B^v = A^v 
\end{equation}
同样的,有
\[
\begin{aligned}
    1 & = (A \to B \wedge C)^v \\
      & = 1 - A^v + A^v \cdot (B \wedge C)^v \\
      & = 1 - A^v + A^v \cdot B^v \cdot C^v 
\end{aligned}
\]
即
\begin{equation}
    A^v \cdot B^v \cdot C^v = A^v
\end{equation}
而
\begin{equation}
    \begin{split}
        (\neg B \to C)^v & = 1 - (\neg B)^v + (\neg B)^v \cdot C^v \\
                         & = 1 - (1 - B^v) + (1 - B^v) \cdot C^v \\
                         & = B^v + C^v - B^v \cdot C^v 
    \end{split}    
\end{equation}
当 $A^v = 1$ 时,由 式(1)和式(2) 可得 $B^v = 1, \; C^v = 1$,此时式(3)成立,
当 $A^v = 0$ 时,由 式(1)和式(2) 无法得出 $B^v, \; C^v$ 的值,此时无法判断式(3)是否成立,
故该逻辑蕴含不成立。\\
3.\\
(1)
\[
\begin{aligned}
    \neg(q \to p) \wedge (r \to \neg s) & = \neg (\neg q \vee p) \wedge (\neg r \vee \neg s) \\
                                        & = q \wedge \neg p \wedge (\neg r \vee \neg s) \mbox{(合取范式)}\\
                                        & = (q \wedge \neg p \wedge \neg r) \vee (q \wedge \neg p \wedge \neg s) \mbox{(析取范式)}
\end{aligned}    
\]
(2)
\[
\begin{aligned}
    \neg p \wedge q \to r & = \neg (\neg p \vee q) \wedge r \\
                          & = p \wedge \neg q \wedge r \mbox{(析取范式)} \\
                          & = (p \wedge \neg q \wedge r) \mbox{(合取范式)} 
\end{aligned}    
\]
(3)\
\[
\begin{aligned}
    \neg (p \vee q) \leftrightarrow p \wedge q & = (\neg (p \vee q) \wedge p \wedge q) \vee (\neg \neg(p \vee q) \wedge \neg (p \wedge q)) \\
                                               & = (p \vee q) \wedge (\neg p \vee \neg q)\mbox{(合取范式)} \\
                                               & = ((p \vee q) \wedge \neg p) \vee ((p \vee q) \wedge \neg q) \\
                                               & = (p \wedge \neg q) \vee (q \wedge \neg p) \vee (p \wedge \neg q) \vee (q \wedge \neg q) \\
                                               & = (q \wedge \neg p) \vee (p \wedge \neg q)\mbox{(析取范式)}
\end{aligned}    
\]
\newpage 
4.\\
(1) 
\begin{table}[h!]
    \begin{center}
      \caption{$p \to p \wedge q$ 真值表}
      \setlength{\tabcolsep}{8mm} {
      \begin{tabular}{|c|c|c|c|} 
        \textbf{$p$} & \textbf{$q$} & \textbf{$p \wedge q$} & \textbf{$p \to p \wedge q$}\\
        \hline
        0 & 0 & 0 & 1\\
        0 & 1 & 0 & 1\\
        1 & 0 & 0 & 0\\
        1 & 1 & 1 & 1\\
      \end{tabular} }
    \end{center}
  \end{table}
\\
从而 $p \to p \wedge q$ 的主合取范式为 
\[  
(\neg p \vee q)
\]
主析取范式为
\[
(\neg p \wedge \neg q) \vee (\neg p \wedge q) \vee (p \wedge q)  
\]
(2)
\begin{table}[h!]
  \begin{center}
    \caption{$p \vee q \to (q \to r)$ 真值表}
    \setlength{\tabcolsep}{8mm} {
    \begin{tabular}{|c|c|c|c|c|c|} 
      \textbf{$p$} & \textbf{$q$} & \textbf{$r$} & \textbf{$p \vee q$} & \textbf{$q \to r$} & \textbf{$p \vee q \to (q \to r)$}\\
      \hline
      0 & 0 & 0 & 0 & 1 & 1\\
      0 & 0 & 1 & 0 & 1 & 1\\
      0 & 1 & 0 & 1 & 0 & 0\\
      0 & 1 & 1 & 1 & 1 & 1\\
      1 & 0 & 0 & 1 & 1 & 1\\
      1 & 0 & 1 & 1 & 1 & 1\\
      1 & 1 & 0 & 1 & 0 & 0\\
      1 & 1 & 1 & 1 & 1 & 1\\
    \end{tabular} }
  \end{center}
\end{table}
\\
从而 $p \vee q \to (q \to r)$ 的主合取范式为
\[
(p \vee \neg q \vee r) \wedge (\neg p \vee \neg q \vee r)  
\]
主析取范式为
\[
(\neg p \wedge \neg q \wedge \neg r) \vee (\neg p \wedge \neg q \wedge r) \vee 
(\neg p \wedge q \wedge r) \vee (p \wedge \neg q \wedge \neg r) \vee (p \wedge \neg q \wedge r)
\vee (p \wedge q \wedge r)  
\]
\newpage 
(3)
\begin{table}[h!]
  \begin{center}
    \caption{$(p \to p \wedge q) \vee r$ 真值表}
    \setlength{\tabcolsep}{8mm} {
    \begin{tabular}{|c|c|c|c|c|c|} 
      \textbf{$p$} & \textbf{$q$} & \textbf{$r$} & \textbf{$p \wedge q$} & \textbf{$p \to p \wedge q$} & \textbf{$(p \to p \wedge q) \vee r$}\\
      \hline
      0 & 0 & 0 & 0 & 1 & 1\\
      0 & 0 & 1 & 0 & 1 & 1\\
      0 & 1 & 0 & 0 & 1 & 1\\
      0 & 1 & 1 & 0 & 1 & 1\\
      1 & 0 & 0 & 0 & 0 & 0\\
      1 & 0 & 1 & 0 & 0 & 1\\
      1 & 1 & 0 & 1 & 1 & 1\\
      1 & 1 & 1 & 1 & 1 & 1\\
    \end{tabular} }
  \end{center}
\end{table}
\\
从而主合取范式为
\[
(\neg p \vee q \vee r)  
\]
主析取范式为
\[
(\neg p \wedge \neg q \wedge \neg r) \vee (\neg p \wedge \neg q \wedge r) \vee 
(\neg p \wedge q \wedge \neg r) \vee (\neg p \wedge q \wedge r) \vee (p \wedge \neg q \wedge r) \vee
(p \wedge q \wedge \neg r) \vee (p \wedge q \wedge r)
\]
\end{document}