\documentclass{article}
\usepackage{xeCJK}
\usepackage{amsmath}

\setCJKmainfont{Microsoft YaHei}
\linespread{1.5}
\setlength{\parindent}{0pt}


\begin{document}
一.\\
(1) $(BC)^+ = \{B, C, A, D, E\}$ \\
(2) $\{A\}, \{B, C\}, \{E\}, \{C, D\}$ \\
(3) 
\[
    F = \{A \to B, A \to C, CD \to E, B \to D, E \to A\}
\]
(4)属于 $1NF$, 对于候选键 $\{B, C\}$, 存在非主属性对其的部分函数依赖 $B \to D$,故
该关系模式不属于 $2NF$\\
(5) 
\[
    \{R1(A, B, C), R2(C, D, E), R3(B, D), R4(E, A)\}
\]
二.\\
(1) 不保持函数依赖,
$ABD$ 上有 $A \to BD$, $BC$ 上有 $B \to C$, $DE$ 上有 $D \to E$ \\
上述合并得到的函数依赖集无法推导出$CE \to A$ \\
(2) 不保持函数依赖,
$ABD$ 上有 $A \to BD$, $BC$ 上有 $B \to C$, $CE$ 上有 $C \to E$ \\
上述合并得到的函数依赖集无法推导出$E \to A$
\end{document}